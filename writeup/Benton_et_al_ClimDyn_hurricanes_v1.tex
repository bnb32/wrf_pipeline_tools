% This is a general template file for the LaTeX package SVJour3
% for Springer journals. Original by Springer Heidelberg, 2010/09/16
%
% Use it as the basis for your article. Delete % signs as needed.
%
% This template includes a few options for different layouts and
% content for various journals. Please consult a previous issue of
% your journal as needed.
%
\RequirePackage{fix-cm}
%
%\documentclass{svjour3}                     % onecolumn (standard format)
%\documentclass[smallcondensed]{svjour3}     % onecolumn (ditto)
\documentclass[smallextended]{svjour3}       % onecolumn (second format)
%\documentclass[a4paper]{article}
%\documentclass[twocolumn]{svjour3}          % twocolumn
%
\smartqed  % flush right qed marks, e.g. at end of proof
%
\usepackage{graphicx}
\usepackage{booktabs}
\usepackage{afterpage}
\setcounter{tocdepth}{4}
\setcounter{secnumdepth}{4}
\newcommand{\myparagraph}[1]{\paragraph{#1}\mbox{}\\\mbox{}\\}
\usepackage{setspace}
\doublespacing
%
% insert here the call for the packages your document requires
%\usepackage{mathptmx}      % use Times fonts if available on your TeX system
%\usepackage{latexsym}
% etc.
%
% please place your own definitions here and don't use \def but
% \newcommand{}{}
%
% Insert the name of "your journal" with
\journalname{Climate Dynamics}
%
\begin{document}

\title{Minor impacts of major volcanic eruptions in
  dynamically-downscaled last millennium ensemble data} %COMMENT: OK?

\thanks{NSF grants AGS1602564 and 1751535 as well as an AWS computing award.} %COMMENT: get correct verbiage

% Grants or other notes about the article that should go on the front
% page should be placed within the \thanks{} command in the title
% (and the %-sign in front of \thanks{} should be deleted)
%
% General acknowledgments should be placed at the end of the article.



\titlerunning{Volcanic effects on hurricanes}

\author{Benton, Brandon N.$^1$ \and Alessi, Marc J.$^1$ \and Herrera, Dimitris$^1$ \and Ault, Toby R.$^1$}

%\authorrunning{Short form of author list} % if too long for running head

\institute{Benton, Brandon N. \at
              %Cornell University \\
	      %Tel.: +123-45-678910\\
	      %Fax: +123-45-678910\\
	      \email{bnb32@cornell.edu}           %  \\
%             \emph{Present address:} of F. Author  %  if needed
              \and
	   Ault, Toby R. \at
	   %Cornell University \\
	   \email{tra38@cornell.edu}
}

\date{$^1$Cornell University \\
Received: date / Accepted: date}
% The correct dates will be entered by the editor

\maketitle

\begin{abstract}
  Thermodynamic and dynamic effects of volcanic eruptions on hurricane
  statistics are examined using long two simulations from the
  Community Earth System Model (CESM) Last Millennium Ensemble
  (LME). The first is an unforced control simulation, wherein all
  boundary conditions were held constant at their 850 CE values. The
  second is a ``fully forced'' simulation with time evolving radiative
  changes from solar, volcanic, solar, and land use changes from 850
  through present. The largest magnitude radiative forcings during
  this time period are the large tropical volcanic eruptions, which
  comprise the focus of this study. Potential and simulated hurricane
  statistics are computed from both the control and forced
  simulations. Potential Intensity is evaluated using model output at
  its native (nominally 2 degree lat/lon) spatial resolution, while
  the weather research and forecasting (WRF) model is used for
  dynamically downscaling a total of 100 control years and an
  additional 100 years following the largest volcanic eruptions in the
  fully forced simulation.  Limitations of the downscaling methodology
  are examined by applying the same approach to historical ERAI
  reanalysis data and comparing the downscaled storm tracks and
  intensities to the IBTrACS database. Results suggest small effects
  are observed in averages over all last millennium eruptions which
  are non-significant in comparison to the control. However, for many
  of the major eruptions, significant reductions are seen in hurricane
  frequency, intensity, and lifetime. Strong evidence is also shown
  for correlation between eruption strength and changes in these
  diagnostics.
%\keywords{Hurricanes \and downscaling \and climatology}
% \PACS{PACS code1 \and PACS code2 \and more}
% \subclass{MSC code1 \and MSC code2 \and more}
\end{abstract}
\keywords{Volcanos \and Hurricanes \and Last Millennium Ensemble \and
  Paleoclimate \and Climate modeling}

\section{Introduction}
\label{intro}
\par
\textbf{Hurricanes threaten human lives and likelihoods, inflict severe damage
to property, and incur billions of dollars in economic losses and
recovery efforts.} These events alone caused $42\%$ of the
catastrophe-insured losses in the United States in the period
$1992-2011$ \cite{hodges2017well}. For example, in 2005 Hurricane Katrina
resulted in $1833$ casualties and a financial loss of over $\$125$
billion \cite{hodges2017well}. In 2012, Hurricane Sandy killed at least
233 people and caused $\$70$ billion in damage
\cite{blake2013tropical}. Moreover, in 2017 Hurricane Harvey displaced more
than $30,000$ people and resulted in $103$ deaths
\cite{murphy2018service}. The same year Hurricane Maria resulted in nearly
$3,000$ deaths \cite{maria_rpt}. Combined, Harvey and Maria caused
more than $\$200$ billion in damage \cite{murphy2018service},
\cite{maria_rpt}. The disruption from these extreme weather events
will likely increase with rising coastal populations and increasing
value of infrastructure in coastal areas
\cite{kerry_tc_clim}. Furthermore, anthropogenic climate change is
expected to increase average sea surface temperatures (SSTs)
\cite{ipcc_2007} and sea level. Accordingly, there is growing interest
in determining if modifications to the incoming flux of solar
radiation could potentially offset key impacts expected to occur from
rising global temperature \cite{REF}. %%irving2019hurricane 
Whether or not
such strategies are pursued, it is critical to understand the
relationship between hurricane statistics and climate responses to
past radiative forcings to help characterize the full
range of plausible future influences on hurricane activity in the
future.
\par


\textbf{The underlying relationship between hurricanes, radiative
  forcing, and climate change remains an area of considerable debate
  and active inquiry \cite{REF,REF,REF}}. Yet, a number of modeling
studies have suggested that, in general, future storms may pose even
more severe threats to human well-being, infrastructure, and the
economy \cite{IPCC2014c}. For example, \cite{REF} and \cite{REF} used
data from the Climate Model Intercomparison 5 (CMIP5) ensemble to
evaluate storm intensity during climate change if their tracks were
similar to those that unfolded over the 21st Century. Such studies
suggest that, in general, the number and intensity of the largest
storms (e.g., category 4 and 5 hurricanes) will increase in a warmer
climate due, primarily, to increases in sea surface temperature
(SSTs). 

\textbf{If global greenhouse gas reduction efforts fail or are
  insufficient in the coming decades, some researchers argue that
  ``solar radiation management'' interventions to the climate system
  may be preferable to allowing global temperatures to increase
  indefinitely \cite{REF,REF,REF}.} %%irving2019hurricane
  Regardless of the
details of how SRM interventions are modeled, their ultimate effect is
to decrease the total amount of sunlight reaching the surface, which
is directly analogous to the effect of stratospheric aerosols from
volcanic eruptions. Therefore volcanic eruptions of the recent past
and last millennium may give us a glimpse of the risks associated with
SRM due to changes in large-scale changes in circulation \cite{REF}.


\textbf{The historical period only provides a few clues about the
  relationship between volcanic forcing and hurricanes, and much less
  about the effect of volcanic eruptions on hurricanes or TCs.}
Nevertheless, modeling studies suggest that a reduction in TC
accumulated energy, TC duration, and lifetime maximum intensity occurs
following a volcanic eruption due to a decrease in SST and increase in
upper tropospheric/lower stratospheric temperature \cite{volc_hurrs2},
all of which decreases TC efficiency \cite{trop_cool}.  There is
evidence that after some eruptions, an asymmetric increase in
stratospheric aerosols occurs in the hemisphere in which the eruption
took place, modifying the sea surface temperature gradient
\cite{asym_forcing}.  This gradient shifts the location of the
Inter-tropical Convergence Zone (ITCZ) to the opposite hemisphere of
the eruption \cite{asym_forcing}, which hinders hurricane development
in the volcano’s own hemisphere due to a decrease in convection and
increase in wind shear.  In fact, after the northern hemisphere
eruptions of Mount Pinatubo (1991) and El Chicon (1982), North
Atlantic TC activity decreased, while TC activity increased following
the southern hemisphere eruption of Agung (1964)
(\cite{volc_hurrs3},\cite{volc_hurrs2}).

\textbf{In the limited insight afforded by the paleoclimate record of
paleotempestology, there seems to be XXXXX.}%DIMITRIS


\textbf{Hurricanes are mesoscale features of the tropical circulation,
  and as such they depend critically on quantities that are typically
  unresolvable in the coarse resolution grid of the CMIP5 generation
  of models, which typically have nominal horizontal resolutions on
  the order of 50-200km.} Overcoming this limitation requires one of
three approaches. The first approach is relatively simple, and entails
calculating thermodynamic metrics like \textit{potential intensity}
(PI) at the native (coarse) resolution of GCM output
\cite{wang,ke_nolan,tang}. Such indices can then be used to infer what
would have happened if hurricanes had been resolved in a given
simulation.  In a recent study \cite{REF} %%Yan2018 
did just than using the
last millennium ensemble (LME) to determine the theoretical effects of
volcanic eruptions during the last 1000 years on hurricane/TC
potential intensity. The authors found a significant relationship
lasting up to 3 years post-eruption, but also ``divergent'' responses
at the mid and high latitudes to the volcanic forcing. While this
approach is computationally efficient, it doesn't explicitly attempt
to \textit{simulate} Hurricanes/TCs.

A second approach has been to use a statistical method to downscale
model output \cite{down_method_ke,cam_down_ke}. Although this approach
is computationally lightweight, allowing it to be used to investigate
long term variability in a fully coupled GCM simulation of the last
millennium \cite{lme_down_ke}, it does not directly simulate the
sub-grid scale features of Hurricanes and TCs.

The third approach employs a regional model to \textit{dynamically}
downscale GCM output \cite{down_21st_gv}. Dynamical
downscaling typically requires high performance computing
infrastructure as well as boundary conditions from the ``parent'' GCM
at six hourly temporal resolution. It is therefore much more
computationally expensive than the other two methods, but it provides
greater insight into the storms that would have occurred in a given
GCM framework if it were run with sufficiently high spatial
resolution. Dynamical downscaling has been widely used to evaluate
hurricane statistics during the 20th and 21st century \cite{REF}, yet
it has not been widely adapted to the last millennium paleoclimate
modeling context.

\textbf{Here we dynamically downscale two members of the ``Last
  Millennium Ensemble'' (LME) \cite{gcm_lme}.} The LME consists of
over two dozen fully forced, and single forcing, experiments from the
period spanning 850 CE to 2005. While monthly data was archived for
most of the members of the LME, two simulations were run with
sufficiently high temporal output to allow for high resolution
dynamical downscaling using a regional model. One of these runs was a
fully forced last millennium simulation and the other was a long
control simulation with time invariant boundary
conditions.

\section{Data and Methods}
\label{methods}
We employed both a control and forced GCM simulation to overcome the limits of relying on historical hurricane records and to separate natural variability from forced behavior. As previously mentioned, the effect of aerosol forcing from volcanic eruptions on hurricane statistics is our primary focus. Both the forced and control simulations are dynamically downscaled using the Weather Research and Forecasting model (WRF) \cite{wrf_tech} and TCs are detected in the downscaled results using the Geophysical Fluid Dynamics Laboratory (GFDL) TC tracking algorithm (TSTORMS). A suite of diagnostics are used to assess the spatial and temporal statistics of hurricanes, as well as to calculate their tracks and extract trends. ERA-Interim (ERAI) reanalysis data is also downscaled and compared to International Best Track Archive for Climate Stewardship (IBTrACS) data to assess accuracy of WRF downscaling and set an uncertainty baseline.
\subsection{Data}
\subsubsection{LME}
We used two simulations from the National Center for Atmospheric Research (NCAR) Last Millennium Ensemble (LME) runs, described in \cite{gcm_lme}. These experiments consist of a control and a forced run, allowing us to assess internal variability as well as the effect of various forcings on hurricane statistics. Both of these simulations were run from 850 to 2005 CE using the Community Earth System Model (CESM) version 1.1, with the Community Atmosphere Model (CAM) version 5. The resolution of the atmosphere and land grids are ${\sim}2^\circ$, and ${\sim}1^\circ$ for ocean and sea ice grids. All LME runs were spun up for 200 years under control conditions prior to 850 CE. The selected runs will be referred to as $LME_{control}$ and $LME_{forced}$. $LME_{forced}$ was forced with the transient evolution of solar intensity, volcanic emissions, greenhouse gases, aerosols, land-use conditions, and orbital parameters. The forcings used in $LME_{forced}$ were the climate forcing reconstructions from phase 5 of the Coupled Intermodel Comparison Project (CMIP5) \cite{gmd-4-33-2011}. $LME_{control}$ was run absent of any of these forcings, thus providing a baseline for the natural climate variability of hurricane statistics. 

\subsubsection{ERAI and IBTRACS}
\label{erai}
We used ERAI and IBTrACS to construct a baseline accuracy and calibration for the downscaling and storm tracking pipeline. ERAI provides a reanalysis dataset starting in 1979 and available until August 2019. We were able to use this dataset in the same way we used GCM data from the LME runs. This allowed us to generate hurricane statistics based on observational climate data. IBTrACS provides observational data for hurricanes for roughly the same time period as ERAI. Thus, comparing the downscaled ERAI results to hurricane tracks and intensities from IBTrACS allowed us to evaluate the accuracy of our approach. Due to the inherently chaotic nature of hurricane genesis exact agreement between ERAI and IBTrACS was not expected. Assessing our approach was the primary objective in comparing ERAI with IBTrACS. ERAI data has resolution on the order of one degree which limits the ability to match individual hurricanes through downscaling. We expected ERAI to capture the observational record for mean climate. This should provide good agreement between downscaled results and overall hurricane statistics seen in IBTrACS. 
\par
ERAI uses four-dimensional variational data assimilation (4DVAR), yielding a significant advantage over reanalysis products using 3DVAR. This improves asynoptic data handling and allows for the influence of an observation to be more strongly controlled by model dynamics \cite{tc_reanal:2}. This data assimilation method is coupled with the ECMWF Integrated Forecast Model (IFS) to extrapolate fields between observations. A detailed description of IBTrACS is provided in \cite{ibtracs} and ERAI is comprehensively discussed in \cite{erai_reanal}. In our work, $6$-hourly ERAI data was downscaled in WRF and compared to IBTrACS for the period $1995-2005$. The comparison was made using the suite of diagnostics described in \ref{diags}. 
\par
In ref. \cite{hodges2017well}, the authors assess how well TCs are represented in reanalysis products. This work used two TC-track matching approaches, referred to as (1) ``direct matching" and (2) ``objective matching". The authors further used several diagnostics in order to compare reanalysis TC tracks to those found in IBTrACS. The objective matching approach, which employs a tracking algorithm similar to TSTORMS, found an agreement of ${\sim}60\%$ with ERAI in the Northern Hemisphere. This agreement also highlights the objective in our own comparison. Exact agreement was not expected, rather the goal was to ensure proper execution of all the computational elements involved in our methodology. 
\subsection{Methods}
\subsubsection{Potential Intensity}
In addition to downscaling, we use the original CESM data to compute potential intensity (PI) fields. This gives us insight into the theoretical effect of eruptions on hurricanes, without the computational overhead of downscaling. Following the thermodynamic analysis in \cite{pi_ke}, we use Equation (\ref{PI_eqn}) to calculate PI:

\begin{equation}
{V_m} \propto \sqrt{\frac{T_s-T_{o}}{T_{o}}(k^{*}-k)},
\label{PI_eqn}
\end{equation}

where \ref{PI_eqn} $V_m$ is the maximum tangential wind speed, $T_s$ is the temperature at the ocean surface, $T_o$ is the outflow temperature at the top of the troposphere, and $k^{*}-k$ is the enthalpy flux (or latent heat flux) at the sea-air interface. The enthalpy flux is given by $c_p(T_{SST}-T_{air})+L(q^{*}-q)$, where $c_p$ is the specific heat capacity at constant pressure, $T_{SST}$ is the sea surface temperature, $T_{air}$ is the temperature of air at the surface, $L$ is the latent heat of vaporization, $q^{*}$ is the saturated specific humidity at the surface, and $q$ is the specific humidity of air at the surface. We are only interested in the fractional difference given by Equation (\ref{dpi}), so we are unconcerned with additional scaling factors.

\begin{equation}
\delta PI = \frac{{V_{m}}^{LME_{forced}}}{{V_{m}}^{LME_{control}}}-1
\label{dpi}
\end{equation}

\subsubsection{Dynamical Downscaling with WRF}
\label{WRF}
As previously mentioned, GCMs are used to generate boundary conditons for RCMs. In this work we used WRF, which is an open-source RCM and is continuously updated based on developments in mesoscale modelling. WRF has a well documented and extensive suite of physics parameterization schemes. This extensive suite provides options suitable for studying hurricane behavior. 
\par
We used WRF vwersion 3.9 (WRFV3.9) \cite{wrf_tech} to dynamically downscale archived data from LME simulations with the physics schemes shown in table \ref{wrf_specs}: (1) WRF single-moment 6-class for micro-physics \cite{mp_phys}, (2) Yonsei University for PBL \cite{pbl_phys}, (3) Kain-Fritsch for convection \cite{cu_phys}, (4,5) rapid radiative transfer model with greenhouse gases for long-wave and short-wave radiation \cite{rad_phys}, (6) Noah for land surface \cite{sfc_phys}, (7) fifth generation mesoscale model for surface layer (\cite{sfclay_phys:1},\cite{sfclay_phys:2},\cite{sfclay_phys:3}), and (8) simple mixed-layer for ocean \cite{ocn_phys}. We also turned on heat and moisture surfaces fluxes (isfflx=1) and modification of exchange coefficients $C_d$ and $C_k$ according to surface winds (isftcflx=1). 
\par
To use data from the LME runs as boundary conditions for WRF, it was necessary to convert the data into a WRF intermediate format. To do so, we employed the routine described in \cite{tech_notes}. We used a WRF domain extending from $130W$ to $15E$ and from the equator to $55N$. This domain allowed tracking of TCs throughout the North Atlantic and after making landfall on the North American Continent. The spatial resolution used in WRF was $30km$. We used adaptive time stepping with a maximum temporal resolution of $240s$.
\par
All dynamical downscaling with WRF was done using the Cheyenne supercomputer built by NCAR. The GCM data was used in parallel WRF simulations using 256 cores at a time. Each downscaling run used approximately 1,500 core hours with pre and post-processing bringing the total to around 2,000. A total of nearly one million core hours were used to complete all the computational work for this project.  

\begin{table}[!tbp]
\centering
\begin{tabular}{lrrr}
\toprule
             Physics Schemes &  Name & Parameter & Value \\ 
\midrule
            (1) Micro-physics &     WSM6 &  mp\_physics & 6 \\  
            (2) PBL &    YSU &  bl\_pbl\_physics &  1 \\    
            (3) Convection &   Kain-Fritsch &  cu\_physics & 1 \\    
            (4) Long-wave radiation &    RRTMG &   ra\_lw\_physics & 4 \\    
            (5) Short-wave radiation &    RRTMG &   ra\_sw\_physics & 4 \\    
            (6) Land surface &   Noah &   sf\_surface\_physics & 2 \\    
            (7) Surface layer &    MM5 &  sf\_sfclay\_physics &  1 \\    
            (8) Ocean &    Mixed-layer &  sf\_ocean\_physics &  1 \\    
\bottomrule
\end{tabular}
\caption{WRF Physics Schemes}
\label{wrf_specs}
\end{table}

\subsubsection{Tracking Tropical Cyclones with TSTORMS}
\label{tstorms}
We used the GFDL developed TSTORMS \cite{tc_algo} TC tracking routine to analyze the results of downscaling. This routine uses minimum pressure and maximum vorticity criteria to identify cyclones. The cyclones are then stored as storms if they satisfy the following conditions for a threshold number of days ($n_{days}$): (1) That the maximum vorticity location is within a threshold radius ($r_{crit}$) of the minimum pressure location, (2) that the core temperature of the cyclone is higher than outside of the core by a threshold difference ($twc_{crit}$) and (3) the difference in vertical distance between pressure levels at $200hPa$ and $1000hPa$ outside and inside the core exceeds a threshold value ($thick_{crit}$). As described in \cite{kerry_clivar} and \cite{tc_algo}, tracking results are sensitive to the particular tracking scheme and threshold values used. However, the latter is responsible for the main difference in tracking scheme results \cite{tc_track}. 
\par
To identify sensitivity to threshold values, we conducted a limited parameter sweep to determine optimal threshold values. We calculated the difference between ERAI downscaled output and IBTrACS data, for each set of parameters, using the diagnostics described in section \ref{diags}. We used the set of parameters that achieved the minimum difference of ${\sim}13.5\%$. This parameter set was $r_{crit} = 1.5^{\circ}$, $twc_{crit} = 1.0^{\circ}C$, $thick_{crit} = 50m$, and $n_{days} = 2$.  
\subsubsection{Diagnostics}
\label{diags}
We estimated a suite of diagnostics to analyze TC tracking data from TSTORMS. These diagnostics consist of storm number vs (1) month, (2) year, (3) latitude, (4) longitude, (5) maximum wind speed, (6) minimum pressure, (7) decay time from maximum wind speed, and (8) decay time from minimum pressure. Additionally, we calculated percentage of storms within (9) May to November, (10) $0-25N$ latitude, (11) $100W-50W$ longitude, (12) $1020hPa-980hPa$ pressure, (13) $0m/s-40m/s$ maximum wind speed, (14) $0-100hrs$ decay time from maximum wind speed, and (15) $0-100hrs$ decay time from minimum pressure. Mean values of the storm number diagnostics and percentage values were used to calculate percentage differences and these differences were averaged over all diagnostics for a composite percentage difference. We refer to the mean difference of diagnostics 1-8 as the total average difference and the mean difference of diagnostics 9-15 as the total percentage difference.  When used initially to assess our methodology these percentage differences were calculated for ERAI vs. IBTrACS, as described in sections \ref{tstorms} and \ref{erai}. 
\par
The diagnostics described above were used as test statistics to evaluate whether volcanic eruptions have a measurable effect on hurricane behavior. These diagnostics were selected in order to assess hurricane behavior across a broad range of characteristics. The diagnostics not only quantify hurricane behavior across the temporal and spatial domain, but also assess more fundamental physical characteristics. In addition, the diagnostics can be used with limited data consisting only of time, location, wind speed, and surface pressure. This presents a versatile and efficient approach to capture both mean climatology and more fine structured hurricane behavior. 
\par
To determine whether volcanic eruptions effect hurricane statistics, we performed two-sample KS-tests for distributions of each of the diagnostics. The two samples tested for each diagnostic came from downscaled $LME_{control}$ and $LME_{forced}$ data. Since $LME_{control}$ does not include volcanic eruptions, agreement with $LME_{control}$ is confirmation of the null hypothesis.   

\subsubsection{Case Studies}
\label{cases}
In this section we present some limited case studies of two few well known hurricanes: (1) Mitch (1998) and (2) Katrina (2005). As previously mentioned, we did not expect the downscaled ERAI to perfectly match IBTrACS. This is due to the inherently chaotic nature of hurricanes, the native resolution of ERAI, and systematic under-estimation of intensities by WRF. However, by increasing the resolution of the WRF simulations we can roughly match some individual hurricanes. Instead of the $30 km$ resolution used for the rest of the downscaled simulations, here we use $10 km$. This significantly increases the computational overhead but allows us to better identify TCs. We handled this constraint by reducing the domain to $100 W$-$20 W$ and $5 N$-$45 N$. We also only simulated the month containing the storm of interest, rather than the entire year. All physics schemes were kept the same as previously described. 

\myparagraph{Hurricane Mitch}
Hurricane Mitch is the second deadliest Atlantic hurricane in recorded history, responsible for over 11,000 deaths in Central America alone due to rain-induced flooding \cite{hellin}.  After forming in the southwestern Caribbean Sea on 22 October 1998, Mitch strengthened into a Category 5 hurricane, attaining a minimum pressure of 905 mb, which is the eighth lowest pressure ever recorded in the Atlantic Hurricane Basin \cite{pasch}.  Mitch turned southward and slowly weakened before making landfall as a minimal hurricane in Honduras.  The nearly stationary movement of the hurricane (4 kt for a week) pulled moisture from the Pacific Ocean and Caribbean into the mountains of Honduras and Nicaragua, resulting in daily rainfalls of over one foot due to orographic lift.  A peak storm total of 75 in. was estimated from satellite-derived methods \cite{hellin}.  After the center of the remnant low emerged over the Gulf of Mexico, it regained tropical storm strengthand made landfall in Florida on October 5 \cite{pasch}.  Mitch accelerated northeast,underwent extra-tropical transition, and dissipated.
\par
The results from downscaling ERAI at 10 km are shown in figure \ref{mitch_tracks} (A) and IBTrACS is shown in figure \ref{mitch_tracks} (B). TSTORMS detects Mitch in the 10 km downscaled ERAI output on October 21, 1998. However, TSTORMS only resolves Mitch until October 25, 1998. We see from the figures that the location of genesis is in good agreeent with IBTrACS but the intensity is underestimated. This is likely why the track is observed to end on October 24. To contrast we show the closest candidate for Mitch from downscaling with 30 km resolution in figure \ref{mitch_tracks} (C). TSTORMS resolved this storm from October 23 to October 25. We see a difference in location as well as intensity variation over the track.  


\begin{figure}[!tbp]
\centering
%\begin{minipage}[b]{0.45\textwidth}
\includegraphics[width=\textwidth]{./figures/Hurricane_Mitch_tracks.eps}
\caption{Hurricane Mitch: (A) ERAI 10 km, (B) IBTRACS, (C) ERAI 30 km}
\label{mitch_tracks}
\end{figure}

\myparagraph{Hurricane Katrina}
Hurricane Katrina is one of the deadliest Atlantic hurricanes and the costliest hurricane in United States history \cite{beven}.  Katrina developed over the central Bahamas on 24 August 2005 and strengthened into a Category 1 hurricane before making landfall in southern Florida.  After briefly weakening to a tropical storm, Katrina strengthened once over the Gulf of Mexico and underwent an eye replacement cycle that doubled the size of the tropical wind field.  A newly defined eyewall developed and the storm underwent rapid intensification, with peak winds reaching 150 kt \cite{beven}.  Katrina made landfall in southeastern Louisiana on 29 August.  Its large wind field drove a storm surge into Mississippi, Alabama, and Lake Pontchartrain, with the latter infamously breaking the levees in New Orleans, leading to widespread destruction and casualties.  Katrina dissipated and was absorbed by a frontal zone as it moved from the Southeast to the Northeast US.
\par
The results from downscaling ERAI are shown in figure \ref{katrina_tracks} (A) and IBTrACS is shown in figure \ref{katrina_tracks} (B). TSTORMS detected Katrina on August 21, 2005 and the track ended August 29, 2005. We see from the figures that again the location is in reasonably good agreement with IBTrACS and the intensity is also in close agreement. To contrast we show the closest candidate for Katrina using $30km$ resolution in figure \ref{katrina_tracks} (C). TSTORMS tracked this storm from September 1, 2005 to September 3, 2005. Not only is the date significantly divergent but the location is as well. Finally, we see huge degradation in the intensity profile as compared to that from $10km$ resolution.    

\begin{figure}[!tbp]
\centering
%\begin{minipage}[b]{0.45\textwidth}
\includegraphics[width=\textwidth]{./figures/Hurricane_Katrina_tracks.eps}
\caption{Hurricane Katrina: (A) ERAI 10 km, (B) IBTRACS, (C) ERAI 30 km}
\label{katrina_tracks}
\end{figure}

\section{Results}
\label{results}
Before delving into the results from downscaling, we look at the much less computationally intensive PI analysis. It is possible to compute the PI field for the entirety of the LME simulations, but it is not feasible to do with dynamical downscaling. Here we present the average potential intensity anomaly for all eruptions and for the strongest eruptions. We also show the PI anomaly for the 10 strongest eruptions and 10 weakest eruptions. PI shows what hurricane behavior should be expected if all hurricanes achieved maximum possible intensity based on thermodynamic conditions of the environment. However, this theoretical intensity is rarely achieved by hurricanes in practice. The results from downscaling are then explored as a check against the theoretical PI predictions. These downscaled results reveal a finer-grained picture of hurricanes and also one more closely aligned with practical outcomes.     
\par
These results consist of comparisons between downscaled output from ERAI and IBTrACS as well as between $LME_{forced}$ and $LME_{control}$. The ERAI vs. IBTrACS comparison provides a baseline of absolute accuracy. The comparison between $LME_{forced}$ and $LME_{control}$ is focused specifically on the effect of aerosol forcing from volcanic eruptions. We downscaled 150 consecutive years of $LME_{control}$ and 100 years of $LME_{forced}$ combined from 2 year runs after 50 separate volcanic eruptions. The reconstruction of these eruptions is described in detail in \cite{erups_recon}.    

\subsection{Potential Intensity}
The PI anomalies ($\delta PI$) for the strongest eruptions are shown in figures \ref{pi_10_avg} and \ref{pi_10_str}. $\delta PI$ for the weakest eruptions is shown in figure \ref{pi_10_wk}. Strength of eruptions were determined by the peak aerosol mass, as shown in figure \ref{erups_plot}. The average fractional PI anomaly for all eruptions is shown in figure \ref{pi_all_avg}. The hatching in figure \ref{pi_all_avg} is based on a p-value threshold of 0.01 for a two-sided t-test at each grid point. This analysis was done for figure \ref{pi_10_avg} as well. The disparity between the two figures supports the notion that an effect on hurricanes is observed only for the largest eruptions. The average decrease in PI for the strongest eruptions is ${\sim}2.2\%$. The average fractional PI anomaly for all eruptions is a decrease of ${\sim}1.0\%$.   

\begin{figure}[!tbp]
\centering
\includegraphics[width=\textwidth]{./figures/PI_diff_50_avg.eps}
\caption{Average PI anomaly for all eruptions. Hatching is based on a p-value threshold of 0.01 for a two-sided t-test at each grid point. We see that any anomalies in the main development region are non-significant and even observe some warming in the North Atlantic.}
\label{pi_all_avg}
\end{figure}

\begin{figure}[!tbp]
\centering
\includegraphics[width=\textwidth]{./figures/PI_diff_10_avg.eps}
\caption{Average PI anomaly for strongest eruptions. All points are below a p-value threshold of 0.01 for a two-sided t-test. Here we see cooling across the Atlantic basin although the main development region sees some warming.}
\label{pi_10_avg}
\end{figure}

\begin{figure}[!tbp]
\centering
\includegraphics[width=\textwidth]{./figures/PI_diff_50.eps}
\caption{PI anomaly for weakest eruptions. Weakest eruptions are those with lowest peak aerosol mass. Here we do not observe any anomalies consistent across all eruptions.}
\label{pi_10_wk}
\end{figure}

\begin{figure}[!tbp]
\centering
\includegraphics[width=\textwidth]{./figures/PI_diff_10.eps}
\caption{PI anomaly for strongest eruptions. Strongest eruptions are those with highest peak aerosol mass. Here we see cooling within the Atlantic basin for all eruptions.}
\label{pi_10_str}
\end{figure}

\begin{figure}[!tbp]
\centering
\begin{minipage}[b]{0.45\textwidth}
\includegraphics[width=\textwidth]{./figures/eruptions_plot.eps}
\caption{Aerosol mass signals for volcanic eruptions 500-2000 C.E. The peak signals shown here are used to determine eruption strength. }
\label{erups_plot}
\end{minipage}
\hfill
\begin{minipage}[b]{0.45\textwidth}
\includegraphics[width=\textwidth]{./figures/power_spectrum.eps}
\caption{$LME_{control}$ SST Power Spectrum. This plot shows that using 100 years of control data is sufficient and in doing so we are not missing any low frequency content.}
\label{spectrum}
\end{minipage}
\end{figure}

\subsection{ERAI vs IBTrACS}
As shown in table \ref{evi_table}, using our suite of diagnostics, we found an overall agreement between ERAI and IBTrACS of ${\sim}86.5\%$, or a composite difference of ${\sim}13.5\%$. As mentioned in section \ref{erai}, $6$-hourly ERAI data downscaled in WRF was compared to IBTrACS for the same time period ($1995-2005$). Diagnostics distributions for both ERAI and IBTrACS are shown in figure \ref{evi_diags}, and tracks for both cases are shown in figure \ref{erai_ibtracs_tracks}. It is worth noting that truncation of the domain in our ERAI simulations contributes to the differences in latitude and longitude peaks seen in figure \ref{evi_diags}.  
\par
We also implemented a rudimentary version of our own track matching algorithm and we saw similar agreement to that in \cite{hodges2017well}. We also saw close agreement comparing results produced by other diagnostics, similar to those described in \ref{diags}. The physics schemes in section \ref{WRF} and threshold values in section \ref{tstorms} were used in response to a self-selected $15\%$ difference threshold imposed between ERAI and IBTrACS, as quantified by our diagnostics suite. 

\begin{figure}[!tbp]
\centering
\includegraphics[width=\textwidth]{./figures/ERAI_vs_IBTRACS_tracks.eps}
\caption{ERAI (A) vs IBTRACS (B) 1995-2005. Here we see good agreement in the location of TC tracks. We note that our WRF domain truncates the ERAI tracks. We also see some underestimation of intensities in downscaled results. Resolvable intensity depends strongly on WRF resolution.}
\label{erai_ibtracs_tracks}
\end{figure}


\begin{figure}[!tbp]
\centering
\includegraphics[width=\textwidth]{./figures/Forced_vs_Control_tracks.eps}
\caption{$LME_{forced}$ all eruptions (A) vs $LME_{control}$ 1000-1100 (B). We see close agreement between forced and control when comparing all simulation years.}
\label{forced_ctrl_tracks}
\end{figure}

\begin{figure}[!tbp]
\centering
\includegraphics[width=\textwidth]{./figures/erai_ibtracs_diags.eps}
\caption{ERAI vs IBTrACS 1995-2005. Here we see good agreement in lifetime and frequency metrics. Location metrics differ mainly due to WRF domain truncation of ERAI tracks. We also see some slight intensity underestimation in ERAI due to WRF resolution.}
\label{evi_diags}
\end{figure}

\begin{table}[!tbp]
\centering
\begin{minipage}[b]{0.45\textwidth}
\begin{tabular}{lrrr}
\toprule
             Averages &         ERAI &      IBTrACS \\ 
\midrule
            month &     7.65 &     8.24 \\  
       yearly num &    31.27 &    34.82 \\   
              lat &    18.23 &    21.39 \\    
              lon &   -80.72 &   -88.38 \\    
     max wind m/s &    30.26 &    34.87 \\    
    min press hPa &   988.56 &   979.65 \\    
           w-life &    45.37 &    44.69 \\    
           p-life &    44.02 &    52.53 \\    
\bottomrule
\end{tabular}
\end{minipage}
\hfill
\begin{minipage}[b]{0.45\textwidth}
\begin{tabular}{lrrr}
\toprule
             Percents &         ERAI &      IBTrACS \\ 
\midrule


             May-Nov &     0.88 &     0.99 \\    
               0-25N &     0.78 &     0.73 \\   
             100-50W &     0.65 &     0.44 \\   
             0-40m/s &     0.97 &     0.70 \\   
         1020-980hPa &     0.81 &     0.62 \\   
        (w) 0-100hrs &     0.94 &     0.91 \\   
        (p) 0-100hrs &     0.94 &     0.87 \\
 \\

\bottomrule
\end{tabular}
\end{minipage}

\noindent\fbox{\parbox{\textwidth}{%
\centering
total average difference: 0.096\\
total percent difference: 0.18\\
composite difference: 0.135}}
\caption{ERAI vs IBTrACS stats}
\label{evi_table}
\end{table}

\subsection{Effect of Eruptions on Hurricane Statistics}
\subsubsection{Average Effect of Eruptions}

In comparing $LME_{control}$ and $LME_{forced}$ we focused on the effect of aerosol forcing from volcanic eruptions. These signals are shown in figure \ref{erups_plot}. We selected 50 eruptions from $LME_{forced}$ and ran WRF for two years after each of the eruptions. $LME_{control}$ was run using WRF for 150 years to give a sufficient sample of natural variability. The $LME_{control}$ run was ensured to have sufficient length by looking at the SST signal in frequency space, as shown in figure \ref{spectrum}. This figure shows no significant low frequency variability is missing from the control sample. The hurricane tracks over 100 years of both $LME_{control}$ and $LME_{forced}$ are shown in figure \ref{forced_ctrl_tracks}.
\par
Distributions of the diagnostics for $LME_{control}$ and $LME_{forced}$, with all 50 eruptions included, are shown in figure \ref{50_erups}. Performing two sample ks-tests on the distributions, along with significance tests on the difference of mean values, shows that the overall effect of all 50 eruptions is consistent with the null hypothesis. That is, the overall effect of all 50 eruptions is consistent with the natural climate variability seen in $LME_{control}$. Figures \ref{ks_all} and \ref{sig_all} show the results of these tests. The ks-tests show a maximum difference between the two samples (D-value), and a probability that the two samples are drawn from the same distribution (P-value). The significance tests show the fraction of the $LME_{control}$ sample which is greater than and less than the mean value of the corresponding $LME_{forced}$ diagnostic. 
\par
Although the aggregate effect of eruptions on hurricanes seems non-significant, we also calculated pearson correlation coefficients on eruption strength and diagnostic changes. The significance of the calculated coefficients can be evaluated by determining the confidence interval for zero correlation. The 90\% confidence interval for zero correlation with all eruptions is $[-0.235,0.235]$. This is the interval of a discrete normal distribution with $N=50$ samples (eruptions) which includes correlation values with probabilities greater than 0.05. Due to the symmetry of the normal distribution the left and right tails outside this interval total a probability of 0.1. Thus, we can say with at least 90\% confidence that yearly number, intensity, and lifetime, correspond with eruption strength. The correlation coefficients are listed in figure \ref{corr_all}.   

\begin{figure}[!tbp]
\centering
\includegraphics[width=\textwidth]{./figures/50_erups_dists.eps}
\caption{$LME_{control}$ vs $LME_{forced}$ with all eruptions. Here we see qualitatively similar profiles for each metric. Notable is the frequency reduction in the forced distributions. }
\label{50_erups}
\end{figure}

\begin{figure}[!tbp]
\centering
\includegraphics[width=\textwidth]{./figures/10_erups_dists.eps}
\caption{$LME_{control}$ vs $LME_{forced}$ with strongest eruptions. Here we again see qualitatively similar profiles for each metric. The frequency reduction here is more pronounced than for the comparison with all eruptions. }
\label{10_erups}
\end{figure}

\begin{table}[!tbp]
\centering
\begin{minipage}[b]{0.45\textwidth}
\begin{tabular}{lrrr}
\toprule
             KS-Tests &     D-value &      P-value\\
\midrule

month & 0.0 & 1.0 \\
yearly num & 0.006 & 1.0 \\
lats & 0.004 & 1.0 \\
lons & 0.0 & 1.0 \\
max wind & 0.006 & 1.0 \\
min press & 0.006 & 1.0 \\
w-life & 0.002 & 1.0 \\
p-life & 0.0 & 1.0 \\

\bottomrule
\end{tabular}
\caption{ks-tests with all eruptions}
\label{ks_all}
\end{minipage}
\hfill
\begin{minipage}[b]{0.45\textwidth}
\begin{tabular}{lrrr}
\toprule
             Sig-Tests & \% greater &  \% less \\
\midrule

month & 0.513 & 0.474 \\
yearly num & 0.435 & 0.565 \\
lats & 0.494 & 0.506 \\
lons & 0.455 & 0.545 \\
max wind & 0.519 & 0.474 \\
min press & 0.513 & 0.487 \\
w-life & 0.565 & 0.435 \\
p-life & 0.506 & 0.494 \\

\bottomrule
\end{tabular}
\caption{significance tests with all eruptions}
\label{sig_all}
\end{minipage}
\end{table}

\begin{table}[!tbp]
\centering
\begin{tabular}{lrrr}
\toprule
             Correlation-Tests &     Correlations \\
\midrule

month & -0.1095 \\
yearly num & -0.2305 \\
lats & 0.0201 \\
lons & 0.2199 \\
max wind & -0.3185 \\
min press & 0.2913 \\
w-life & -0.0901 \\
p-life & -0.2753 \\

\bottomrule
\end{tabular}
\caption{correlations with all eruptions}
\label{corr_all}
\end{table}

\subsubsection{Effect of Strongest Eruptions}
Distributions of diagnostics with only the 10 strongest eruptions included are shown in figure \ref{10_erups}. Tables showing the results of ks-tests and significance tests on the strongest eruptions are in figures \ref{ks_10} and \ref{sig_10}. We see in the significance table that the $LME_{forced}$ mean values for yearly number, intensity, and lifetime suggest we should reject the null hypothesis at only the $70\%-80\%$ confidence limit. These signals show that for the 10 largest eruptions the yearly number, intensity, and lifetimes are reduced. Interestingly, the eruptions with the largest net effects are 1213 (8th strongest) and 1815 (3rd strongest). This clearly demonstrates that other factors are at work besides amount of aerosol forcing. Both the 1213 and 1815 eruptions have ${\sim}13\%$ total average difference from $LME_{control}$. Tables shows the respective significance tests are shown in figures \ref{sig_1213} and \ref{sig_1815}.   

\begin{table}[!tbp]
\centering
\begin{minipage}[b]{0.45\textwidth}
\begin{tabular}{lrrr}
\toprule
             KS-Tests & D-value & P-value \\
\midrule
month & 0.004 & 1.0 \\
yearly num & 0.018 & 1.0 \\
lats & 0.036 & 0.9 \\
lons & 0.038 & 0.86 \\
max wind & 0.024 & 1.0 \\
min press & 0.048 & 0.6 \\
w-life & 0.014 & 1.0 \\
p-life & 0.012 & 1.0 \\

\bottomrule
\end{tabular}
\caption{ks-tests with strong eruptions}
\label{ks_10}
\end{minipage}
\hfill
\begin{minipage}[b]{0.45\textwidth}
\begin{tabular}{lrrr}
\toprule
             Sig-Tests & \% greater & \% less \\
\midrule

month & 0.461 & 0.513 \\
yearly num & 0.584 & 0.351 \\
lats & 0.487 & 0.513 \\
lons & 0.318 & 0.682 \\
max wind & 0.773 & 0.227 \\
min press & 0.286 & 0.714 \\
w-life & 0.675 & 0.325 \\
p-life & 0.708 & 0.292 \\

\bottomrule
\end{tabular}
\caption{significance tests with strong eruptions}
\label{sig_10}
\end{minipage}
\end{table}


\begin{table}[!tbp]
\centering
\begin{minipage}[b]{0.45\textwidth}
\begin{tabular}{lrrr}
\toprule
             Sig-Tests & \% greater &  \% less \\

\midrule

month & 0.63 & 0.357 \\
yearly num & 0.812 & 0.169 \\
lats & 0.747 & 0.253 \\
lons & 0.708 & 0.292 \\
max wind & 1.0 & 0.0 \\
min press & 0.0 & 1.0 \\
w-life & 0.896 & 0.104 \\
p-life & 0.981 & 0.019 \\

\bottomrule
\end{tabular}
\caption{sig-tests for 1213 eruption}
\label{sig_1213}
\end{minipage}
\hfill
\begin{minipage}[b]{0.45\textwidth}
\begin{tabular}{lrrr}
\toprule
             Sig-Tests & \% greater &  \% less \\
\midrule

month & 0.513 & 0.481 \\
yearly num & 0.883 & 0.084 \\
lats & 0.325 & 0.675 \\
lons & 0.195 & 0.805 \\
max wind & 0.831 & 0.169 \\
min press & 0.058 & 0.942 \\
w-life & 0.896 & 0.104 \\
p-life & 0.942 & 0.058 \\

\bottomrule
\end{tabular}
\caption{sig-tests for 1815 eruption}
\label{sig_1815}
\end{minipage}
\end{table}



\section{Summary and Discussion}
\label{discuss}
In this work we have explored the effect of volcanic eruptions in the past millennium on hurricane climatology. To do this we first validated our approach of downscaling CESM data with WRF by comparing results of ERAI downscaling with IBTrACS data. We also performed a parameter search for our cyclone tracking algorithm in order to achieve high accuracy and to understand sensitivity to selected parameters. We then compared the results of downscaling our control data from CESM with forced data from CESM, where we focused on the years in the forced data which bounded the volcanic eruptions. We found that the aggregate effect of eruptions is consistent with the null hypothesis: the control case. However, we see evidence that sufficiently strong eruptions do result in lower annual hurricane count, reduced intensity, and shorter lifetimes. This evidence is in the form of ks and significance tests on diagnostic distributions, as well as correlations between strength and changes in the mean values of these diagnostics. PI analysis also supports these conclusions. 
\par
Although we see moderate correlation between eruption strength and certain diagnostic measures, it is not necessarily true that stronger eruptions have a larger effect on hurricane statistics. This presents a direction for further investigation. In this vein, one could look at an ensemble of higher resolution GCM simulations on one or two of the strongest volcanic eruptions. This eruption profile will be simulated both in the climate conditions during the historical eruption as well as under future climate change conditions. An ensemble average or simulations with perturbed initial conditions will allow us to home in on the sole effect of aerosol forcing. This will also allow us to explore the question of whether downscaling introduced any unknown biases. An ensemble under future climate change conditions will allow us to explore the interplay of large aerosol forcing and strong anthropogenic forcing.    

%\begin{acknowledgements}
%If you'd like to thank anyone, place your comments here
%and remove the percent signs.
%\end{acknowledgements}

\bibliographystyle{spphys}
\bibliography{hurrclim} 

\end{document}
% end of file template.tex
