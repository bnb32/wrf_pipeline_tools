%\begin{filecontents*}{example.eps}
%%!PS-Adobe-3.0 EPSF-3.0
%%%BoundingBox: 19 19 221 221
%%%CreationDate: Mon Sep 29 1997
%%%Creator: programmed by hand (JK)
%%%EndComments
%gsave
%newpath
%  20 20 moveto
%  20 220 lineto
%  220 220 lineto
%  220 20 lineto
%closepath
%2 setlinewidth
%gsave
%  .4 setgray fill
%grestore
%stroke
%grestore
%\end{filecontents*}
%
\RequirePackage{fix-cm}
%
%\documentclass{svjour3}                     % onecolumn (standard format)
%\documentclass[smallcondensed]{svjour3}     % onecolumn (ditto)
%\documentclass[smallextended]{svjour3}       % onecolumn (second format)
%\documentclass[twocolumn]{svjour3}          % twocolumn
%
\documentclass[a4paper]{article}
%\usepackage[affil-it]{authblk}
%\smartqed  % flush right qed marks, e.g. at end of proof
%
\usepackage{graphicx}
\usepackage{booktabs}
\usepackage{afterpage}
%
% \usepackage{mathptmx}      % use Times fonts if available on your TeX system
%
% insert here the call for the packages your document requires
%\usepackage{latexsym}
% etc.
%
% please place your own definitions here and don't use \def but
% \newcommand{}{}
%
% Insert the name of "your journal" with
%\journalname{Climate Dynamics}
%
\begin{document}

\title{Methods for dynamical downscaling and analysis of last millennium hurricane climatology}

%\thanks{Grants or other notes
%about the article that should go on the front page should be
%placed here. General acknowledgments should be placed at the end of the article.}

%\subtitle{Do you have a subtitle?\\ If so, write it here}

%\titlerunning{Assessing Hurricane climatology}        % if too long for running head

\author{Benton, Brandon N. \and Ault, Toby R.}

%\email{bnb32@cornell.edu}
%\thanks{Corresponding Author} 
%\affiliation{Department of Earth and Atmospheric Science, Cornell University, Ithaca}

%\author{Ault, Toby R.}%
%\email{tobyault@gmail.edu} 
%\affiliation{Department of Earth and Atmospheric Science, Cornell University, Ithaca}

\maketitle

%\authorrunning{Short form of author list} % if too long for running head

%\institute{Benton, Brandon N. \at Cornell University \\
%              \email{bnb32@cornell.edu} \and
%           Ault, Toby R. \at Cornell University \\
%              \email{tobyault@gmail.com}
%}

%\date{Received: date / Accepted: date}
% The correct dates will be entered by the editor


%\maketitle

\begin{abstract}
In this work we explore the effect of volcanic eruptions over the last millennium on hurricane genesis and behavior. We use data from CESM simulations of the last millennium forced with aerosol deposition reconstructed from ice cores. This data is downscaled using WRF and hurricanes are detected and tracked in the downscaled output. This methodology is validated through downscaling ERAI and comparing to IBTrACS. The hurricane climatology from the forced simulations are compared to climatology from simulations performed without aerosol forcing. Potential Intensity is evaluated along with an extensive suite of higher order diagnostics. Small effects are seen in averages over all last millennium eruptions which are consistent with the null hypothesis. However, for many of the largest eruptions, significant reductions are seen in hurricane frequency, intensity, and lifetime. We also see strong evidence for correlation between eruption strength and changes in these diagnostics.   
%\keywords{Hurricanes \and downscaling \and climatology}
% \PACS{PACS code1 \and PACS code2 \and more}
% \subclass{MSC code1 \and MSC code2 \and more}
\end{abstract}

\section{Introduction}
\label{intro}
\par
Recent decades have seen widespread and costly damage caused by tropical cyclones. Tropical cyclones alone caused $42\%$ of the United States catastrophe-insured losses in the period $1992-2011$ \cite{tc_reanal:1}, and Hurricane Katrina resulted in a death toll of $1833$ people with a financial loss of over $\$125$ billion \cite{tc_reanal:1}. The disruption from these extreme weather events will only increase with rising coastal populations and increasing value of infrastructure in coastal areas \cite{kerry_tc_clim}. It is expected that hurricane statistics will change in response to projected changes in climate in significant ways, and understanding these changes is essential in order to safeguard against future destruction. 
\par
Historical records are not extensive enough to support trend analysis, are susceptible to biases due to changes in detection instrumentation, and are limited by lack of aerial observation capability in the past \cite{kerry_clivar}. On the other hand, analysing historical records is but one approach to understanding the relationship between hurricane statistics and forcing. 
\par
Modelling the response of hurricane behavior to climate changes has been explored extensively. Global circulation models (GCMs) and regional climate models (RCMs) are both used in this context. GCMs by themselves have numerous limitations for capturing hurricane statistics. Typical resolutions used in GCM simulations, on the order of degrees, are much too large to generate or observe hurricanes. Lowering the resolution sufficiently requires enormous computational resources and simulation time. Much of the modelling effort so far has focused primarily on capturing accurate current climate conditions which are used as boundary conditions for RCMs to assess hurricane statistics \cite{kerry_clivar}. A major limitation of this approach is the difficulty of separating natural variability from forced responses.
\par
Additionally, overall climate forcing depends on a combination of many separate effects. For instance, increased aerosols and increased greenhouse gases in the atmosphere can have competing effects on sea surface and surface air temperatures which might produce different hurricane statistics. To first order, aerosols reduce insolation thereby reducing surface temperatures and GHGs increase surface temperatures by trapping infrared surface emissions. Understanding the effect of the overall forcing is more tractable when broken up into the effect of individual processes on hurricane behavior. 
\par
Volcanic eruptions are significant sources of aerosols and studying eruptions thus provides insight into how aerosol forcing effects hurricanes. How aerosols effect hurricanes has significant bearing on a number of important questions. These range from how pollution might effect hurricanes, to how to approach adaptation challenges. Because of the corresponding reduction in insolation from aerosols studying eruptions could provide key insight into geoengineering risks. 
\par
Volcanic eruptions can also be roughly viewed as having the equal but negative effect of GHGs. Thus, understanding hurricane statistics in past eruptions provides an understanding of responses to general climate forcing. Hurricane statistics after volcanic eruptions have not been explored in depth and investigation is restricted by the limited number of eruptions in the historical record. However, this make eruptions a particularly rich area for climate model based exploration. 
\par 
There has been significant interest in downscaling global circulation data to simulate hurricanes. In \cite{cam_down_ke} data from CAM is used for downscaling in an environment subjected to artificial increases in carbon dioxide. The downscaling method used is described in \cite{down_method_ke}, and is a statistical approach rather than dynamical like WRF. Long term variability is explored over the last millennium in \cite{lme_down_ke}, which uses a fully coupled GCM as in our work. The downscaling method is again statistical and the focus is on decadal and multi-decadal variability of land-falling hurricanes. In \cite{down_21st_gv} there is a specific focus on downscaling 21st century climate conditions. Our work differs from these experiments by using a fully coupled GCM to drive dynamical downscaling, with a focus on the effect of volcanic eruptions on hurricane statistics over the last millennium.
\par
This paper is organized as follows. We start with a brief summary of the models and methods used, in section \ref{methods}. This section covers both the GCM simulations and WRF downscaling, in addition to the suite of diagnostics employed in our analysis. Here we also discuss the ERAI and IBTrACS data sets, used to validate our methodology, as well as the TSTORMS tracking algorithm used to analyse WRF output. We present our results in section \ref{results} and summarize our conclusions, along with possible directions for future work in section \ref{discuss}.     

\section{Models and Methods}
\label{methods}
In this work we employ both a control and forced GCM simulation to overcome the limits of relying on historical records and to separate natural variability from forced behavior. As mentioned, the effect of aerosol forcing from volcanic eruptions on hurricane statistics is of primary focus. Both the forced and control simulations are dynamically downscaled using the Weather Research and Forecasting model (WRF) and tropical cyclones (TCs) are detected in the downscaled results using the GFDL TC tracking algorithm (TSTORMS). A suite of diagnostics are used to assess the spatial and temporal hurricane statistics and extract climatological trends. ERA-Interim (ERAI) reanalysis data is also downscaled and compared to International Best Track Archive for Climate Stewardship (IBTrACS) data to assess accuracy in WRF downscaling and set an uncertainty baseline.
\subsection{GCMs}
In this work we use two simulations from the NCAR Last Millennium Ensemble (LME) runs, described in \cite{gcm_lme}. Briefly, these experiments consist of a control and a forced run, allowing us to assess internal variability as well as the effect of various forcings on hurricane statistics. Both of these simulations were run from 850 to 2005 CE using the Community Earth System Model (CESM) version 1.1, with the Community Atmosphere Model (CAM) version 5. The resolution of the atmosphere and land grids are $\sim2^\circ$ and $\sim1^\circ$ for ocean and sea ice grids. All LME runs were spun up for 200 years under control conditions prior to 850 CE. In the following the selected runs will be referred to as LME_{control} and LME_{forced}. LME_{forced} was forced with the transient evolution of solar intensity, volcanic emissions, greenhouse gases, aerosols, land-use conditions, and orbital parameters. The forcings used in LME_{forced} were climate forcing reconstructions from phase 5 of the Coupled Intermodel Comparison Project (CMIP5) \cite{gmd-4-33-2011}. LME_{control} was run absent of any of these forcings thus providing a baseline for the background of hurricane statistics natural climate variability. 

\subsection{Potential Intensity}
Before looking at the results of downscaling we use the raw CESM data to compute potential intensity (PI) fields. This gives us insight into the theoretical effect of eruptions on hurricanes, without the computational overhead of downscaling. Following the thermodynamic analysis in \cite{pi_ke}, we use equation \ref{PI_eqn} to calculate PI.

\begin{equation}
{V_m} \propto \sqrt{\frac{T_s-T_{o}}{T_{o}}(k^{*}-k)}
\label{PI_eqn}
\end{equation}

In equation \ref{PI_eqn} $V_m$ is the maximum tangential wind speed, $T_s$ is the temperature at the ocean surface, $T_o$ is the outflow temperature at the top of the troposphere, and $k^{*}-k$ is the enthalpy flux, or latent heat flux, at the sea-air interface. The enthalpy flux is given by $c_p(T_{SST}-T_{air})+L(q^{*}-q)$, where $c_p$ is the specific heat at constant pressure, $T_{SST}$ is the sea surface temperature, $T_{air}$ is the temperature of air at the surface, $L$ is the latent heat of vaporization, $q^{*}$ is the saturated specific humidity at the surface, and $q$ is the specific humidity of air at the surface. We are only interested in the fractional difference given by equation \ref{dpi}, so we are unconcerned with additional scaling factors.

\begin{equation}
\delta PI = {V_{m}}^{LME_{forced}}/{V_{m}}^{LME_{control}}-1
\label{dpi}
\end{equation}


\subsection{WRF}
\label{WRF}
We used WRFV3.9 for the dynamical downscaling with the following physics schemes: WSM6 for micro-physics \cite{mp_phys}, YSU for PBL \cite{pbl_phys}, Kain-Fritsch for convection \cite{cu_phys}, RRTMG for long-wave and short-wave radiation \cite{rad_phys}, Noah for land surface \cite{sfc_phys}, MM5 for surface layer (\cite{sfclay_phys:1},\cite{sfclay_phys:2},\cite{sfclay_phys:3}), and simple mixed-layer for ocean \cite{ocn_phys}. We also turned on heat and moisture surfaces fluxes (isfflx=1) and used the scheme which modifies exchange coefficients $C_d$ and $C_k$ according to surface winds (isftcflx=1). To convert GCM data to use as boundary data for WRF we employed the conversion routine described in \cite{tech_notes}. We use a WRF domain extending from $130W$ to $15E$ and from the equator to $55N$. The spatial resolution used in the WRF downscaling was $30km$. We used adaptive time stepping with a maximum temporal resolution of $240s$.

\subsection{TSTORMS}
\label{tstorms}
The TC tracking routine developed by GFDL  \cite{tc_algo}, called TSTORMS, was used to analyse the results of downscaling. This routine uses minimum pressure and maximum vorticity criteria to select candidate cyclones. The cyclones are then stored as storms if they satisfy the following conditions for a threshold number of days ($n_{days}$): (1) That the maximum vorticity location is within a threshold radius ($r_{crit}$) of the minimum pressure location, (2) that the core temperature of the cyclone is higher than outside of the core by a threshold difference ($twc_{crit}$) and (3) the difference in vertical distance between pressure levels at $200hPa$ and $1000hPa$ outside and inside the core exceeds a threshold value ($thick_{crit}$). As described in \cite{kerry_clivar} and \cite{tc_algo}, tracking results are sensitive to the particular tracking scheme and threshold values used. However, the latter is responsible for the main difference in tracking scheme results \cite{tc_track}. 
\par
To identify sensitivity to threshold values, we conducted a limited parameter sweep to determine optimal threshold values. We calculated the difference between ERAI downscaled output and IBTrACS data, for each set of parameters, using the diagnostics described in section \ref{diags}. We used the set of parameters which achieved the minimum difference of $\sim14\%$. This parameter set was $r_{crit} = 1.3^{\circ}$, $twc_{crit} = 1.25^{\circ}C$, $thick_{crit} = 50m$, and $n_{days} = 2$.  

\subsection{ERAI and IBTrACS}
\label{erai}
We used ERAI and IBTrACS to construct a baseline accuracy and calibration for the downscaling and storm tracking pipeline. ERAI uses four-dimensional variational data assimilation (4DVAR), presenting a significant advantage over reanalysis products using 3DVAR. This improves asynoptic data handling and allows for the influence of an observation to be more strongly controlled by model dynamics \cite{tc_reanal:2}. This data assimilation method is coupled with the ECMWF Integrated Forecast Model (IFS) to extrapolate fields between observations. A detailed description of IBTrACS is provided in \cite{ibtracs} and ERAI is comprehensively discussed in \cite{erai_reanal}. In our work $6$-hourly ERAI data was downscaled in WRF and compared to IBTrACS for the same arbitrarily selected time period ($1995-2005$). The comparison was made using the suite of diagnostics described in \ref{diags}. In \cite{tc_reanal:1} an extensive study was performed of how well TCs are represented in reanalysis products. This work made use of two track matching approaches, referred to as "direct matching" and "objective matching", along with several diagnostics similar to the ones we selected, in order to compare reanalysis TC tracks to those found in IBTrACS. The objective matching approach, which employs a tracking algorithm similar to TSTORMS, found agreement of $\sim60\%$ with ERAI in the Northern Hemisphere. With a quick rudimentary implementation of our own track matching algorithm we saw similar agreement. We also saw close agreement comparing results produced by other diagnostics, similar to those described in \ref{diags}. The physics schemes in section \ref{WRF} and threshold values in section \ref{tstorms} were used in response to a self-selected $15\%$ difference threshold imposed between ERAI and IBTrACS, as quantified by our diagnostics suite.      

\subsection{Diagnostics}
\label{diags}
We estimated a suite of diagnostics to analyse cyclone tracking data from TSTORMS. These diagnostics consist of storm number vs (1) month, (2) year, (3) latitude, (4) longitude, (5) maximum wind speed, (6) minimum pressure, (7) decay time from maximum wind speed, and (8) decay time from minimum pressure. Additionally, we calculated percentage of storms within (9) May to November, (10) $0-25N$ latitude, (11) $100W-50W$ longitude, (12) $1020hPa-950hPa$ pressure, (13) $0m/s-40m/s$ maximum wind speed, (14) $0-100hrs$ decay time from maximum wind speed, and (15) $0-100hrs$ decay time from minimum pressure. Mean values of the storm number diagnostics and percentage values were used to calculate percentage differences and these differences were averaged over all diagnostics for a composite percentage difference. We refer to the mean difference of diagnostics 1-8 as the total average difference and the mean difference of diagnostics 9-15 as the total percentage difference.  When used initially to assess our methodology these percentage differences were calculated for ERAI vs. IBTrACS, as described in sections \ref{tstorms} and \ref{erai}. 
\par
Two-sample KS-tests were also performed for distributions of the same diagnostics described above. These diagnostics were selected in order to assess hurricane behavior across a broad range of characteristics. The diagnostics quantify hurricane behavior across the temporal and spatial domain, and also assess more fundamental physical characteristics. In addition, the diagnostics can be used with limited data consisting only of time, location, wind speed, and pressure. This presents a versatile and efficient approach to capture both mean climatology and more fine structured hurricane behavior.       

\section{Results}
\label{results}
Our results consist of comparisons between downscaled output from ERAI and IBTrACS as well as between LME_{forced} and LME_{control}. The ERAI vs. IBTrACS comparison provides a baseline of absolute accuracy. The comparison between LME_{forced} and LME_{control} is focused specifically on the effect of aerosol forcing from volcanic eruptions. We downscaled 150 consecutive years of LME_{control} and 100 years of LME_{forced} combined from 2 year runs after 50 separate volcanic eruptions. The reconstruction of these eruptions is described in detail in \cite{erups_recon}.    

\subsection{ERAI vs IBTrACS}
We found an overall agreement between ERAI and IBTrACS of $\sim86\%$, or a composite difference of $\sim14\%$, as shown in figure \ref{evi_table}. As mentioned in section \ref{erai}, $6$-hourly ERAI data was downscaled in WRF and compared to IBTrACS for the same arbitrarily selected time period ($1995-2005$). Agreement was evaluated using the diagnostics described in section \ref{diags}. Diagnostics distributions for both ERAI and IBTrACS are shown in figure \ref{evi_diags}, and tracks for both cases are shown in figures \ref{erai_tracks} and \ref{ibtracs_tracks}. It is worth noting that truncation of the domain in our ERAI simulations contributes to the differences in latitude and longitude peaks seen in figure \ref{evi_diags}.  


\begin{figure}[!tbp]
\centering
\includegraphics[width=\textwidth]{./figures/erai_1995_2005_34_tracks.pdf}
\caption{ERAI 1995-2005}
\label{erai_tracks}
\end{figure}

\begin{figure}[!tbp]
\centering
\includegraphics[width=\textwidth]{./figures/ibtracs_1995_2005_tracks.pdf}
\caption{IBTrACS 1995-2005}
\label{ibtracs_tracks}
\end{figure}

\begin{figure}[!tbp]
\centering
\includegraphics[width=\textwidth]{./figures/erai_ibtracs_diags.pdf}
\caption{ERAI vs IBTrACS 1995-2005 (ERAI in blue, IBTrACS in orange)}
\label{evi_diags}
\end{figure}


\begin{figure}[!tbp]
\centering
\begin{minipage}[b]{0.45\textwidth}
\begin{tabular}{lrrr}
\toprule
             Averages &         ERAI &      IBTrACS \\ 
\midrule
            month &     7.63 &     8.24 \\  
              lat &    17.76 &    21.56 \\    
              lon &   -82.62 &   -88.30 \\    
     max wind m/s &    30.03 &    34.87 \\    
    min press hPa &   992.95 &   979.65 \\    
           w-life &    40.22 &    44.69 \\    
           p-life &    39.31 &    52.53 \\    
             year &     2000.07 &  2000.35 \\
       yearly num &    35.64 &    34.82 \\   
\bottomrule
\end{tabular}
\end{minipage}
\hfill
\begin{minipage}[b]{0.45\textwidth}
\begin{tabular}{lrrr}
\toprule
             Percents &         ERAI &      IBTrACS \\ 
\midrule


             May-Nov &     0.81 &     0.99 \\    
               0-25N &     0.80 &     0.73 \\   
             100-50W &     0.60 &     0.41 \\   
             0-40m/s &     0.97 &     0.70 \\   
         1020-950hPa &     1.00 &     0.84 \\   
        (w) 0-100hrs &     0.96 &     0.91 \\   
        (p) 0-100hrs &     0.97 &     0.87 \\
 \\
 \\

\bottomrule
\end{tabular}
\end{minipage}

\noindent\fbox{\parbox{\textwidth}{%
\centering
total average difference: 0.10\\
total percent difference: 0.19\\
composite difference: 0.14}}
\caption{ERAI vs IBTrACS stats}
\label{evi_table}
\end{figure}

\subsection{Potential Intensity}
The potential intensity anomalies ($\delta PI$) for the strongest eruptions are shown in figures \ref{pi_10_avg} and \ref{pi_10_str}. $\delta PI$ for the weakest eruptions is shown in figure \ref{pi_10_wk}. The average fractional difference for all eruptions is shown in figure \ref{pi_all_avg}. The hatching seen in figure \ref{pi_all_avg} is based on a p-value threshold of 0.05 for a two-sided t-test at each grid point. This analysis was done for figure \ref{pi_10_avg} as well. The disparity between the two figures supports the notion that an effect on hurricanes is seen only for the largest eruptions. The average decrease in PI for the strongest eruptions is $\sim2.2\%$. The average fractional difference for all eruptions is a decrease of $\sim1.0\%$.   

\begin{figure}[!tbp]
\centering
\includegraphics[width=\textwidth]{./figures/PI_diff_50_avg.pdf}
\caption{Average PI anomaly for all eruptions. Hatching is based on a p-value threshold of 0.05 for a two-sided t-test at each grid point.}
\label{pi_all_avg}
\end{figure}

\begin{figure}[!tbp]
\centering
\includegraphics[width=\textwidth]{./figures/PI_diff_10_avg.pdf}
\caption{Average PI anomaly for strongest eruptions. All points are below a p-value threshold of 0.05 for a two-sided t-test.}
\label{pi_10_avg}
\end{figure}

\begin{figure}[!tbp]
\centering
\includegraphics[width=\textwidth]{./figures/PI_diff_50.pdf}
\caption{PI anomaly for weakest eruptions}
\label{pi_10_wk}
\end{figure}

\begin{figure}[!tbp]
\centering
\includegraphics[width=\textwidth]{./figures/PI_diff_10.pdf}
\caption{PI anomaly for strongest eruptions}
\label{pi_10_str}
\end{figure}



\subsection{Average Effect of Eruptions}
In comparing LME_{control} and LME_{forced} we focused on the effect of aerosol forcing from volcanic eruptions. These signals are shown in figure \ref{erups_plot}. We selected 50 eruptions in the last millennium and ran WRF for two years after each of the eruptions. LME_{control} was run in WRF for 150 years to give a sufficient sample of natural variability. The LME_{control} run was ensured to have sufficient length by looking at the SST signal in frequency space, as shown in figure \ref{spectrum}.

\begin{figure}[!tbp]
\centering
\begin{minipage}[b]{0.45\textwidth}
\includegraphics[width=\textwidth]{./figures/eruptions_plot.pdf}
\caption{Aerosol signals 500-2000 C.E.}
\label{erups_plot}
\end{minipage}
\hfill
\begin{minipage}[b]{0.45\textwidth}
\includegraphics[width=\textwidth]{./figures/power_spectrum.pdf}
\caption{LME_{control} SST Power Spectrum}
\label{spectrum}
\end{minipage}
\end{figure}


Distributions of the diagnostics for LME_{control} and LME_{forced}, with all 50 eruptions included, are shown in figure \ref{50_erups}. Performing two sample ks-tests on the distributions, along with significance tests on the difference of mean values, shows that the overall effect of all 50 eruptions is consistent with the null hypothesis. Figures \ref{ks_all} and \ref{sig_all} show the results of these tests. The ks-tests show a maximum difference between the two samples (D-value), and a probability that the two samples are drawn from the same distribution (P-value). The significance tests show the fraction of the LME_{control} sample which is greater than and less than the mean value of the corresponding LME_{forced} diagnostic. Although the aggregate effect of eruptions on hurricanes seems insignificant, we also calculated pearson correlation coefficients on eruption strength and diagnostic changes. The significance of the calculated coefficients can be evaluated by determining the confidence interval for zero correlation. The 90\% confidence interval for zero correlation with all eruptions is $[-0.235,0.235]$. Thus, we can say with at least 90\% confidence that yearly number, intensity, and lifetime, are correspond with eruption strength. The correlation coefficients are listed in figure \ref{corr_all}.   

\begin{figure}[!tbp]
\centering
\includegraphics[width=\textwidth]{./figures/50_erups_dists.pdf}
\caption{LME_{control} vs LME_{forced}, with all eruptions (LME_{control} in orange, LME_{forced} in blue)}
\label{50_erups}
\end{figure}

\begin{figure}[!tbp]
\centering
\includegraphics[width=\textwidth]{./figures/10_erups_dists.pdf}
\caption{LME_{control} vs LME_{forced}, with strongest eruptions (LME_{control} in orange, LME_{forced} in blue)}
\label{10_erups}
\end{figure}

\begin{figure}[!tbp]
\centering
\begin{minipage}[b]{0.45\textwidth}
\begin{tabular}{lrrr}
\toprule
             KS-Tests &     D-value &      P-value\\
\midrule

month & 0.167 & 0.99 \\
yearly num & 0.167 & 0.76 \\
lats & 0.109 & 0.88 \\
lons & 0.083 & 0.99 \\
max wind & 0.08 & 1.0 \\
min press & 0.086 & 0.95 \\
w-life & 0.107 & 1.0 \\
p-life & 0.107 & 1.0 \\

\bottomrule
\end{tabular}
\caption{ks-tests with all eruptions}
\label{ks_all}
\end{minipage}
\hfill
\begin{minipage}[b]{0.45\textwidth}
\begin{tabular}{lrrr}
\toprule
             Sig-Tests & \% greater &  \% less \\
\midrule

month & 0.513 & 0.487 \\
yearly num & 0.506 & 0.494 \\
lats & 0.422 & 0.578 \\
lons & 0.455 & 0.545 \\
max wind & 0.487 & 0.513 \\
min press & 0.519 & 0.468 \\
w-life & 0.526 & 0.474 \\
p-life & 0.519 & 0.481 \\

\bottomrule
\end{tabular}
\caption{significance tests with all eruptions}
\label{sig_all}
\end{minipage}
\end{figure}

\begin{figure}[!tbp]
\centering
\begin{tabular}{lrrr}
\toprule
             Correlation-Tests &     Correlations \\
\midrule

month & -0.0008 \\
yearly num & -0.2913 \\
lats & -0.1373 \\
lons & 0.1857 \\
max wind & -0.3375 \\
min press & 0.2827 \\
w-life & -0.2363 \\
p-life & -0.3499 \\

\bottomrule
\end{tabular}
\caption{correlations with all eruptions}
\label{corr_all}
\end{figure}


\subsection{Effect of Strongest Eruptions}
Distributions of diagnostics with only the 10 strongest eruptions included are shown in figure \ref{10_erups}. Tables showing the results of ks-tests and significance tests on the strongest eruptions are in figures \ref{ks_10} and \ref{sig_10}. We see in the significance table that the LME_{forced} mean values for yearly number, intensity, and lifetime are consistent with a rejection of the null hypothesis with $70\%-80\%$ confidence. These signals show that for the 10 largest eruptions the yearly number, intensity, and lifetimes are reduced. Interestingly, the eruptions with the largest net effects are 1213 (8th strongest) and 1815 (3rd strongest). This clearly shows that other factors are at work besides amount of aerosol forcing. Both the 1213 and 1815 eruptions have $\sim13\%$ total average difference from LME_{control}. Tables shows the respective significance tests are shown in figures \ref{sig_1213} and \ref{sig_1815}.   

\begin{figure}[!tbp]
\centering
\begin{minipage}[b]{0.45\textwidth}
\begin{tabular}{lrrr}
\toprule
             KS-Tests & D-value & P-value \\
\midrule

month & 0.167 & 0.99 \\
yearly num & 0.433 & 0.0 \\
lats & 0.127 & 0.74 \\
lons & 0.167 & 0.48 \\
max wind & 0.1 & 0.95 \\
min press & 0.271 & 0.01 \\
w-life & 0.107 & 1.0 \\
p-life & 0.143 & 0.92 \\

\bottomrule
\end{tabular}
\caption{ks-tests with strong eruptions}
\label{ks_10}
\end{minipage}
\hfill
\begin{minipage}[b]{0.45\textwidth}
\begin{tabular}{lrrr}
\toprule
             Sig-Tests & \% greater & \% less \\
\midrule

month & 0.39 & 0.604 \\
yearly num & 0.714 & 0.286 \\
lats & 0.506 & 0.494 \\
lons & 0.396 & 0.604 \\
max wind & 0.779 & 0.221 \\
min press & 0.305 & 0.695 \\
w-life & 0.714 & 0.286 \\
p-life & 0.708 & 0.292 \\


\bottomrule
\end{tabular}
\caption{significance tests with strong eruptions}
\label{sig_10}
\end{minipage}
\end{figure}


\begin{figure}[!tbp]
\centering
\begin{minipage}[b]{0.45\textwidth}
\begin{tabular}{lrrr}
\toprule
             Sig-Tests & \% greater &  \% less \\

\midrule

month & 0.63 & 0.357 \\
yearly num & 0.812 & 0.169 \\
lats & 0.747 & 0.253 \\
lons & 0.708 & 0.292 \\
max wind & 1.0 & 0.0 \\
min press & 0.0 & 1.0 \\
w-life & 0.896 & 0.104 \\
p-life & 0.981 & 0.019 \\

\bottomrule
\end{tabular}
\caption{sig-tests for 1213 eruption}
\label{sig_1213}
\end{minipage}
\hfill
\begin{minipage}[b]{0.45\textwidth}
\begin{tabular}{lrrr}
\toprule
             Sig-Tests & \% greater &  \% less \\
\midrule

month & 0.513 & 0.481 \\
yearly num & 0.883 & 0.084 \\
lats & 0.325 & 0.675 \\
lons & 0.195 & 0.805 \\
max wind & 0.831 & 0.169 \\
min press & 0.058 & 0.942 \\
w-life & 0.896 & 0.104 \\
p-life & 0.942 & 0.058 \\

\bottomrule
\end{tabular}
\caption{sig-tests for 1815 eruption}
\label{sig_1815}
\end{minipage}
\end{figure}



\section{Summary and Discussion}
\label{discuss}
In this work we have explored the effect of volcanic eruptions in the past millennium on hurricane climatology. To do this we first validated our approach of downscaling CESM data with WRF by comparing results of ERAI downscaling with IBTrACS data. We also performed a parameter search for our cyclone tracking algorithm in order to achieve highest accuracy and to understand sensitivity to selected parameters. We then compared the results of downscaling our control data from CESM with forced data from CESM, where we focused on the years in the forced data which bounded the volcanic eruptions. We found that the aggregate effect of eruptions is consistent with the null hypothesis. However, we see evidence that sufficiently strong eruptions do result in lower annual hurricane count, reduced intensity, and shorter lifetimes. This evidence is in the form of ks and significance tests on diagnostic distributions, as well as correlations between strength and changes in the mean values of these diagnostics. Potential Intensity analysis also supports these conclusions. 
\par
Although we see moderate correlation between eruption strength and certain diagnostic measures, it is not necessarily true that stronger eruptions have a larger effect on hurricane statistics. This presents a direction for further investigation. We would like to further explore the effect of individual eruptions on hurricanes and to further reduce the influence of other forcing factors. In this vein, we plan to look at an ensemble of higher resolution GCM simulations on one or two of the strongest volcanic eruptions. This eruption profile will be simulated both in the climate conditions during the historical eruption as well as under future climate change conditions. An ensemble average or simulations with perturbed initial conditions will allow us to home in on the sole effect of aerosol forcing. This will also allow us to explore the question of whether downscaling introduced any unknown biases. An ensemble under future climate change conditions will allow us to explore the interplay of large aerosol forcing and strong anthropogenic forcing. Furthermore, the data product generated from the downscaling work described above will be used in drought analysis of the northern contiguous United States.   

%\begin{acknowledgements}
%If you'd like to thank anyone, place your comments here
%and remove the percent signs.
%\end{acknowledgements}

% BibTeX users please use one of
%\bibliographystyle{spbasic}      % basic style, author-year citations
\bibliographystyle{spmpsci}      % mathematics and physical sciences
%\bibliographystyle{spphys}       % APS-like style for physics
\bibliography{hurrclim.bib}   % name your BibTeX data base

\end{document}

